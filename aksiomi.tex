\documentclass[12pt,a4paper]{book}

\usepackage{etoolbox}
\providetoggle{long}
\settoggle{long}{true}
%\settoggle{long}{false}

\newcommand{\komentar}[1]{\textcolor{red}{[#1]}} %for displaying red texts

%\textheight=21cm
%\textwidth=16cm
%\oddsidemargin=0.2cm
%\newcommand{\spc}{{\vskip 0.15cm}{\noindent}}
%
\usepackage{fullpage}
\usepackage{fancyhdr}

\usepackage{enumerate}
%\pagestyle{fancy}
%\fancyfoot[ROF,REF]{\tiny Pripravil Martin Milani\v c.}

\usepackage[slovene]{babel} % slovenske nastavitve (naslovi, deljenje besed ...)
\usepackage[utf8]{inputenc}
\usepackage{amsfonts,amssymb}
\usepackage{graphicx}
\usepackage{amsmath,amssymb,amsthm}
\usepackage{epsf}
\usepackage{epsfig}
\usepackage{graphics}
\usepackage{color}
%\usepackage{times}
%\usepackage{txfonts}
\def\P {{\cal P}}
\def\ali {{~\vee~}}
\def\inn {{~\wedge~}}
\def\sledi {{~\Rightarrow~}}
\def\brez {{\,\setminus\,}}
\def\cee {{~\Leftrightarrow~}}
\def\zgled{\noindent{\bf\color{blue} Zgled: }}
\def\kz{{\hfill{\color{blue}$\blacktriangle$}}}% konec zgleda
%\parskip=7pt
%
%\textwidth=17cm\textheight=22cm\parindent=15pt\parskip=5pt
%\oddsidemargin=-5mm  \headheight=0pt \pagestyle{plain}
\newtheorem*{trditev}{Trditev}
\newtheorem*{izrek}{Izrek}
\newtheorem*{problem}{Naloga}
\newtheorem*{lema}{Lema}
\newtheorem*{posledica}{Posledica}
\newtheorem*{definicija}{Definicija}

\def\proofname{Dokaz}

\begin{document}

\section*{Aksiomi teorije množic (po Edertonu)}

\begin{enumerate}
  \item Aksiom ekstenzionalnosti (enakost množic)
  $$\forall A\, \forall B\, [\forall x \,(x\in A \cee x\in B) \sledi A = B]$$
  
  \item Aksiom o prazni množici
  $$\exists B\, \forall x\, (x\not \in B)$$

  \item Aksiom o paru
    $$\forall u\, \forall v\, \exists B\, \forall x\, [x\in B \cee x = u\textrm{ ali }x = v]$$

  \item Aksiom o uniji
    $$\forall A\, \exists B\, \forall x\, [x\in B \cee (\exists b\in B) x\in b]$$

  \item Aksiom o potenčni množici  
    $$\forall a\, \exists B\, \forall x [x\in B \cee x\subseteq a]$$

  \item Aksiomi o podmnožicah
   
  Za vsako formulo $\varphi$ o množicah, ki ne vsebuje črke $B$, 
  je naslednji izraz aksiom:  
$$\forall t_1\,\cdots \forall t_k\,\forall c\,\exists B \,\forall x \,[x\in B \cee x\in c \inn \varphi]$$

  \item Aksiom neskončnosti
$$\exists A\,[\emptyset \,\in A \inn (\forall a\in A)\, a^+\in A]$$
(Pri tem je $a^+ = a\cup \{a\}$.)

\item Aksiom izbire
$$(\forall \textrm{ relacijo }R)(\exists \textrm{ funkcija }F)(F\subseteq R \inn {\cal D}(F) = {\cal D}(R))$$

\item Aksiomi o zamenjavi 

Za vsako formulo $\varphi(x,y)$, ki ne vsebuje črke $B$, 
je naslednji izraz aksiom:
$$\forall t_1\,\cdots \forall t_k\,\forall A\,[(\forall x\in A)\,\forall y_1\, \forall y_2\, (\varphi(x,y_1) \inn \varphi(x,y_2) \sledi y_1 = y_2)$$
$$\sledi \exists B\, \forall y\, [y\in B \cee (\exists x\in A)\,\varphi(x,y)]$$

\item Aksiom regularnosti
$$(\forall A\neq \emptyset)\,(\exists m\in A)\,(m\cap A = \emptyset)$$
\end{enumerate}

\section*{Aksiomi teorije množic (po Dugundjiju)}

\begin{enumerate}
  \item Aksiom individualnosti: $(x\in A) \inn (x = y) \sledi y\in A$
   
  \item Aksiom o formaciji razredov = Aksiom o podmnožicah po Endertonu
  
  \item Aksiom o prazni množici

\item Aksiom o paru

  \item Aksiom o uniji
  
  \item Aksiom o potenčni množici

  \item Aksiom o zamenjavi: slika vsake preslikave je množica.

  \item Aksiomi (``Sifting''): Presek vsake množice z razredom je množica.

\item Aksiom regularnosti

  \item Aksiom neskončnosti

\item Aksiom izbire
\end{enumerate}

\end{document}
